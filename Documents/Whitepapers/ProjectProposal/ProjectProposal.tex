\documentclass[a4paper]{article}

\usepackage[T1]{fontenc}
\usepackage[adobe-utopia]{mathdesign}
\usepackage[protrusion=true,expansion=true]{microtype}
\usepackage{xcolor}

\definecolor{darkblue}{rgb}{0,0,0.5}

\usepackage[colorlinks=true,
        urlcolor=darkblue,
        anchorcolor=darkblue,
        linkcolor=darkblue,
        citecolor=darkblue,
        pdfauthor={Georg Jung},
        pdfkeywords={educational tools, pseudo-code, virtual machine,
          parser, interpreter},
        pdftitle={Psudo-code virtual machine},
        pdfsubject={Bachelor topic proposal for the faculty for MET,
          the German University in Cairo GUC
          (http://met.guc.edu.eg/)}]{hyperref}
\usepackage{url}

\author{Georg Jung}
\title{Pseudo-code virtual machine. A project proposal}

\begin{document}

\maketitle

\begin{abstract}
  Teaching and student interaction depends on a large number of menial
  tasks, such as material preparation, setting up of assignments,
  quizzes, exercises, and exams, tractability of example/exercise
  problems, correct weighting of problems towards the course
  objectives and difficulty level, etc. Further, electronically
  available examples, exercise and experimentation opportunities, and
  tutorials promise to be of tremendous help specifically within the
  mature fields of basic teaching. In this project, various tools for
  in classroom and off classroom teaching and administrative work will
  be created. Their creation will require creativity, good overview to
  understand complex, but repetitive tasks, and a good sense for
  useful and interesting experimentation.

  One of the biggest hurdles for GUC engineering students in their
  first semester is to understand the mechanics of simple
  algorithms. The learning curve of an actual programming language
  such as Java is steep, so it is not suitable for starting to teach
  the concepts of algorithms. Pseudo code on the other hand is
  abstract, no execution environment is available, and the only
  feedback the student receives is the marks on his or her
  assignments.

  To really understand the ideas of algorithms, experimentation with
  immediate feedback is extremely helpful. In this task, a visualizing
  interpreter for a modified pseudo-code will be built, that allows
  for entering, editing, and step-by-step executing of mini-code while
  observing the effects of statements and control-flow as they happen.
\end{abstract}

\end{document}

%%% Local Variables: 
%%% mode: latex
%%% TeX-master: t
%%% End: 